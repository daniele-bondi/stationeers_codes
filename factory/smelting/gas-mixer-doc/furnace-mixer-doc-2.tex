% arara: lualatex: {shell: 1}
\documentclass{article}
\usepackage{geometry}[margin=1in]
\usepackage{booktabs}
\usepackage{hyperref}
\usepackage{subcaption}
\usepackage{minted}
\usepackage{multicol}
\usepackage{parcolumns}
\usepackage{mathpazo}
\usepackage{amsmath}
\usepackage{amssymb}
\usepackage{mathtools}
\usepackage{xfrac}
\usepackage{pgfplots}
\pgfplotsset{compat=1.17}
\usepgflibrary{fpu}

\newcommand\pgfmathparseFPU[1]{\begingroup%
\pgfkeys{/pgf/fpu,/pgf/fpu/output format=fixed}%
\pgfmathparse{#1}%
\pgfmathsmuggle\pgfmathresult\endgroup}%

%\newcommand\pgfmathsetmacroFPU[2]{%
%\pgfmathparseFPU{#2}
%\def{#1}{\pgfmathresult}
%}

\newcommand{\tightmathT}{%
    \setlength{\abovedisplayskip}{0pt}
    \setlength{\abovedisplayshortskip}{0pt}
}
\newcommand{\tightmathB}{%
    \setlength{\belowdisplayskip}{0pt}
    \setlength{\belowdisplayshortskip}{0pt}
}
\newcommand{\tightmath}{%
    \tightmathT%
    \tightmathB%
}

\DeclareMathOperator{\CDiox}{\mathrm{CO}_2}

\def\R{8.314}
\def\vF{1000}
\def\tC{230}
\def\tH{2300}

\newcommand{\surf}{%
    \addplot3[
        surf,
        samples=15,
        samples y=15,
        opacity=0.5,
        domain=0:\nF,
        domain y=\tC:\tH,
        variable=\nR,
    ] 
    {(\tT*\nT-\tF*(\nF-\nR))/y};
}
\newcommand{\Surf}[4]{%
    \pgfmathsetmacro{\pF}{#1}
    \pgfmathsetmacro{\tF}{#2}
    \pgfmathsetmacro{\pT}{#3}
    \pgfmathsetmacro{\tT}{#4}
    \begin{tikzpicture}[scale=0.5]
        \begin{axis}[
                title={$p_F=\pF$, $t_F=\tF$, $p_T=\pT$, $t_T=\tT$}, 
                xlabel=$n_R$, ylabel=$t_I$, zlabel=$n_I$,
                view={-135}{45},
                ymin=\tC, ymax=\tH,
                zmin=0,
            ]
            \surf{}
        \end{axis}
    \end{tikzpicture}
}
\newcommand{\Full}[4]{%
    \pgfmathsetmacro{\pF}{#1}
    \pgfmathsetmacro{\tF}{#2}
    \pgfmathsetmacro{\pT}{#3}
    \pgfmathsetmacro{\tT}{#4}
    \pgfmathparseFPU{\pF/\R/\tF*\vF}\pgfmathsetmacro{\nF}{\pgfmathresult}
    \pgfmathparseFPU{\pT/\R/\tT*\vF}\pgfmathsetmacro{\nT}{\pgfmathresult}
    \pgfmathsetmacro{\rG}{0}
    \pgfmathparseFPU{\nF-(\nT*(\tT-\tH))/(\tF-\tH)}\pgfmathsetmacro{\rR}{\pgfmathresult}
    \pgfmathparseFPU{\nF-(\nT*(\tT-\tC))/(\tF-\tC)}\pgfmathsetmacro{\rB}{\pgfmathresult}

    \begin{tikzpicture}[scale=0.5]
        \begin{axis}[
                title={$p_F=\pF$, $t_F=\tF$, $p_T=\pT$, $t_T=\tT$}, 
                xlabel=$n_R$, ylabel=$t_I$, zlabel=$n_I$,
                view={-135}{45},
                ymin=\tC, ymax=\tH,
                zmin=0,
            ]
            \addplot3[
                surf,
                samples=15,
                samples y=15,
                opacity=0.5,
                domain=0:\nF,
                domain y=\tC:\tH,
                variable=\nR,
            ] 
            {(\tT*\nT-\tF*(\nF-\nR))/y};
            \addplot3+[
                no markers,
                thick,
                color=red,
                samples=30,
                samples y=0,
                domain={max(\rG, max(\rR, \rB))}:\nF,
                domain y=\tC:\tH,
                variable=\nR,
            ]
            ({\nR},{(\tT*\nT-\tF*(\nF-\nR))/(\nT-\nF+\nR)},{\nT-\nF+\nR});
            \filldraw[green]
            (
                {\rG},
                {(\tT*\nT-\tF*\nF)/(\nT-\nF)},
                {\nT-\nF}
            ) circle (2pt) node {};
            \filldraw[red]
            (
                {\rR},
                {\tH},
                {\nT-(\nT*(\tT-\tH))/(\tF-\tH)}
            ) circle (2pt) node {};
            \filldraw[blue]
            (
                {\rB},
                {\tC},
                {\nT-(\nT*(\tT-\tC))/(\tF-\tC)}
            ) circle (2pt) node {};
        \end{axis}
    \end{tikzpicture}
}

\begin{document}

\section{Ideal gas mixer}

\subsection{Utility functions}

\begin{multicols}{2}
    \noindent
    For emptying the input pipe $I$ into the furnace
    \columnbreak
    \begin{minted}{text}
        fillFurnace:
        yield
        s Furnace SettingInput 100
        l x IAnalyzer TotalMoles
        brgtz -3
        s Furnace SettingInput 0
        j ra
    \end{minted}
\end{multicols}

\begin{multicols}{2}
    \noindent
    And for emptying the furnace into the filtration system, which filters all
    $\CDiox$ back into the input pipe.
    \columnbreak
    \begin{minted}{text}
        emptyFurnace:
        yield
        s Furnace SettingOutput 100
        l x Furnace TotalMoles
        brgtz -3
        s Furnace SettingOutput 0
        j ra
    \end{minted}
\end{multicols}

\subsection{Mixing algorithm}

Considering a system of only one gas $g$ with
a hot source $H$ of temperature $t_H$ and
a cold source $C$ of temperature $t_C$,
we can calculate an optimal formulation for bringing a furnace $F$
(with volume $v_F=1000$, initial pressure $p_F$ and initial temperature $t_F$)
to a desired pressure $p_T$ and temperature $t_T$.
This is accomplished by removing an amount $n_R$ from the furnace and/or adding
an amount $n_I$ at a specific temperature $t_I$, where $I$ is composed from
amounts $n_H$ and $n_C$ from the $H$ and $C$ sources.
Several gas-law derived equations constrain this process:
\begin{gather}
    t_T n_T = t_F(n_F-n_R)+t_I n_I \label{eq:1} \\
    t_I n_I = t_H n_H+t_C n_I \label{eq:2} \\
    n_T = n_F-n_R+n_I \label{eq:3}
\end{gather}
Where $n_R$, $t_I$ and $n_I=n_H+n_C$ are to be determined.
Note the constraint for pressures and temperatures in this system
for specific volumes or an arbitrary volume $M$:
\begin{gather*}
    0\le n_R\le n_F \\
    t_C\le t_M\le t_H,\quad
    0\le n_M,\quad
    0\le p_M\le 6\textrm{e}5
\end{gather*}
Because $n_I$ is potentially unbounded, solving equation \ref{eq:1} for $n_I$
provides a surface bounded by $0\le n_R\le n_F$ and $t_C\le t_I\le t_H$
\[
    n_I = \frac{t_T n_T-t_F(n_F-n_R)}{t_I}.
\]
\Surf{4000}{500}{9000}{700}










%\begin{figure}
    %\begin{center}
        %\begin{subfigure}{0.33\textwidth}
            %\begin{tikzpicture}[scale=0.5]
                %\Surf{6000}{600}{21000}{900} % green example
            %\end{tikzpicture}
            %\caption{$H$ and $C$ mixture added.}
        %\end{subfigure}
        %\vspace{1em}

        %\begin{subfigure}{0.33\textwidth}
            %\begin{tikzpicture}[scale=0.5]
                %\Surf{6000}{600}{5000}{800} % red example
            %\end{tikzpicture}
            %\caption{Volume removed, $H$ added.}
        %\end{subfigure}
        %\begin{subfigure}{0.33\textwidth}
            %\begin{tikzpicture}[scale=0.5]
                %\Surf{15000}{900}{6000}{600} % blue example
            %\end{tikzpicture}
            %\caption{Volume removed, $C$ added.}
        %\end{subfigure}
    %\end{center}
    %\caption{%
        %Example bounded solutions.
        %Surface belonging to equations \ref{eq:1} and \ref{eq:2}.
        %Curve belonging to the additional constraint of equation \ref{eq:3}.
    %}
%\end{figure}

%\begin{multicols}{2}
%\raggedcolumns
%\noindent
%\columnbreak

%\begin{minted}{text}
%\end{minted}
%\end{multicols}

\end{document}
