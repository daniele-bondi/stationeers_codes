% arara: lualatex: {shell: 1}
\documentclass{article}
\usepackage{geometry}[margin=1in]
\usepackage[T1]{fontenc}
\usepackage{booktabs}
\usepackage{hyperref}
\usepackage{subcaption}
\usepackage{minted}
%\usepackage{multicol}
\usepackage{parcolumns}
\usepackage{mathpazo}
\usepackage{amsmath}
%\usepackage{amssymb}
%\usepackage{mathtools}
\usepackage{xfrac}
\usepackage{pgfplots}
\pgfplotsset{compat=1.17}
\usepgflibrary{fpu}

\newcommand\pgfmathparseFPU[1]{
    \begingroup%
        \pgfkeys{/pgf/fpu,/pgf/fpu/output format=fixed}
        \pgfmathparse{#1}
        \pgfmathsmuggle\pgfmathresult
    \endgroup
}

\newcommand{\tightmathT}{%
    \setlength{\abovedisplayskip}{0pt}
    \setlength{\abovedisplayshortskip}{0pt}
}
\newcommand{\tightmathB}{%
    \setlength{\belowdisplayskip}{0pt}
    \setlength{\belowdisplayshortskip}{0pt}
}
\newcommand{\tightmath}{%
    \tightmathT%
    \tightmathB%
}

\DeclareMathOperator{\CDiox}{\mathrm{CO}_2}

\begin{document}

\def\GasConstant{8.314}
\def\vF{1000}
\def\tC{230}
\def\tH{2300}

\newcommand{\setupparams}[4]{%
    \pgfmathsetmacro{\pF}{#1}
    \pgfmathsetmacro{\tF}{#2}
    \pgfmathparseFPU{\pF/\tF*\vF/\GasConstant)}\pgfmathsetmacro{\nF}{\pgfmathresult}
    \pgfmathsetmacro{\pT}{#3}
    \pgfmathsetmacro{\tT}{#4}
    \pgfmathparseFPU{\pT/\tT*\vF/\GasConstant)}\pgfmathsetmacro{\nT}{\pgfmathresult}
    \pgfmathsetmacro{\rG}{0}
    \pgfmathparseFPU{\nF-(\nT*(\tT-\tH))/(\tF-\tH)}\pgfmathsetmacro{\rR}{\pgfmathresult}
    \pgfmathparseFPU{\nF-(\nT*(\tT-\tC))/(\tF-\tC)}\pgfmathsetmacro{\rB}{\pgfmathresult}
}
\newcommand{\surf}{%
    \addplot3[%
        surf,
        samples=15,
        samples y=15,
        opacity=0.5,
        domain=0:\nF,
        domain y=\tC:\tH,
        variable=\nR,
    ]
    {(\tT*\nT-\tF*(\nF-\nR))/y};
}
\newcommand{\curve}{%
    \addplot3[%
        thick,
        color=red,
        samples=30,
        samples y=0,
        domain={max(\rG, max(\rR, \rB))}:\nF,
        domain y=\tC:\tH,
        variable=\nR,
    ]
    ({\nR},{(\tT*\nT-\tF*(\nF-\nR))/(\nT-\nF+\nR)},{\nT-\nF+\nR});
}
\newcommand{\Surf}[5][1.0]{%
    \setupparams{#2}{#3}{#4}{#5}
    \begin{tikzpicture}[scale=#1]
        \begin{axis}[
                title={$p_F=\pF$, $t_F=\tF$, $p_T=\pT$, $t_T=\tT$},
                xlabel=$n_R$, ylabel=$t_I$, zlabel=$n_I$,
                view={-135}{45},
                ymin=\tC, ymax=\tH,
                zmin=0,
            ]
            \surf%
            \curve%
        \end{axis}
    \end{tikzpicture}
}
\newcommand{\Full}[5][1.0]{%
    \setupparams{#2}{#3}{#4}{#5}
    \begin{tikzpicture}[scale=#1]
        \begin{axis}[
                title={$p_F=\pF$, $t_F=\tF$, $p_T=\pT$, $t_T=\tT$},
                xlabel=$n_R$, ylabel=$t_I$, zlabel=$n_I$,
                view={-135}{45},
                ymin=\tC, ymax=\tH,
                zmin=0,
            ]
            \surf%
            \curve%
            \filldraw[green]
            (
                {\rG},
                {(\tT*\nT-\tF*\nF)/(\nT-\nF)},
                {\nT-\nF}
            ) circle (2pt) node {};
            \filldraw[red]
            (
                {\rR},
                {\tH},
                {\nT-(\nT*(\tT-\tH))/(\tF-\tH)}
            ) circle (2pt) node {};
            \filldraw[blue]
            (
                {\rB},
                {\tC},
                {\nT-(\nT*(\tT-\tC))/(\tF-\tC)}
            ) circle (2pt) node {};
        \end{axis}
    \end{tikzpicture}
}

\section{Ideal gas mixer}

\subsection{Utility functions}

\begin{parcolumns}{2}
    \colchunk[1]{%
        \noindent
        For emptying the input pipe $I$ into the furnace
    }\colchunk[2]{%
        \begin{minted}{text}
            fillFurnace:
            yield
            s Furnace SettingInput 100
            l x IAnalyzer TotalMoles
            brgtz -3
            s Furnace SettingInput 0
            j ra
        \end{minted}
    }
\end{parcolumns}

\begin{parcolumns}{2}
    \colchunk[1]{%
    %\noindent
        And for emptying the furnace into the filtration system, which filters all
        $\CDiox$ back into the input pipe.
    }\colchunk[2]{%
        \begin{minted}{text}
            emptyFurnace:
            yield
            s Furnace SettingOutput 100
            l x Furnace TotalMoles
            brgtz -3
            s Furnace SettingOutput 0
            j ra
        \end{minted}
    }
\end{parcolumns}

\subsection{Mixing algorithm}

Considering a system of only one gas $g$ with
a hot source $H$ of temperature $t_H$ and
a cold source $C$ of temperature $t_C$,
we can calculate an optimal formulation for bringing a furnace $F$
(with volume $v_F=1000$, initial pressure $p_F$ and initial temperature $t_F$)
to a desired pressure $p_T$ and temperature $t_T$.
This is accomplished by removing an amount $n_R$ from the furnace and/or adding
an amount $n_I$ at a specific temperature $t_I$, where $I$ is composed from
amounts $n_H$ and $n_C$ from the $H$ and $C$ sources.
Several gas-law derived equations constrain this process:
\begin{gather}
    t_T n_T = t_F(n_F-n_R)+t_I n_I \label{eq:surface} \\
    n_T = n_F-n_R+n_I \label{eq:constraint} \\
    t_I n_I = t_H n_H+t_C n_I \label{eq:added}
\end{gather}
Where $n_R$, $t_I$ and $n_I=n_H+n_C$ are to be determined.
Note the constraint for pressures and temperatures in this system
for specific volumes or an arbitrary volume $M$:
\begin{gather*}
    t_C\le t_M\le t_H,\quad
    0\le n_M,\quad
    0\le n_R\le n_F.
\end{gather*}
Solving \autoref{eq:surface} for $n_I$ provides a surface bounded in two dimensions
by $0\le n_R\le n_F$ and $t_C\le t_I\le t_H$, but where $0\le n_I$ is potentially unbounded
(\autoref{fig:surface}).
\[
    (n_R,t_I,f):\quad
    f(n_R,t_I) = n_I = \frac{t_T n_T+t_F(n_R-n_F)}{t_I}.
\]
\autoref{eq:constraint} further restricts potential solutions.
Given a satisfactory $n_R$, $n_I=n_T-n_F+n_R$.
Then we instead solve \autoref{eq:surface} for $t_I$.
As a result, these restricted solutions lie within a curve embedded within the surface.
\[
    (n_R,h,g):\quad
    g(n_R) = n_I = n_T-n_F+n_R,\quad
    h(n_R) = t_I = \frac{t_T n_T-t_F(n_F-n_R)}{n_T-n_F+n_R}
\]
\begin{figure}
    \begin{center}
        \Surf{4000}{500}{9000}{700}
    \end{center}
    \caption[]{%
        The $(n_R,t_I,f)$ solution surface
        and $(n_R,h,g)$ embedded curve.
    }
    \label{fig:surface}
\end{figure}
With respect to $t_I$, $h$ is monotone decreasing, but monotone increasing with respect to $n_R$.
Thus minimizing with respect to $n_I$, considering the constraints, is a matter of solving for where the embedded
curve intersects with either $n_R=0$, $t_I=t_H$, or $t_I=t_C$.







\begin{figure}
    \begin{center}
        \begin{subfigure}{0.48\textwidth}
            \Full[.7]{6000}{600}{21000}{900} % green example
            \caption{$H$ and $C$ mixture added.}
        \end{subfigure}
        \vspace{1em}

        \begin{subfigure}{0.48\textwidth}
            \Full[.7]{6000}{600}{5000}{800} % red example
            \caption{Volume removed, $H$ added.}
        \end{subfigure}
        \begin{subfigure}{0.48\textwidth}
            \Full[.7]{15000}{900}{6000}{600} % blue example
            \caption{Volume removed, $C$ added.}
        \end{subfigure}
    \end{center}
    \caption{%
        Example solutions.
    }
\end{figure}

%\begin{multicols}{2}
%\raggedcolumns
%\noindent
%\columnbreak

%\begin{minted}{text}
%\end{minted}
%\end{multicols}

\end{document}
