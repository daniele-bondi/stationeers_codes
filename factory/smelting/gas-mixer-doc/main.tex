% arara: lualatex: {shell: 1}
\documentclass{article}
\usepackage{geometry}[margin=1in]
\usepackage{booktabs}
\usepackage{hyperref}
\usepackage{minted}
\usepackage{multicol}
\usepackage{parcolumns}
\usepackage{mathpazo}
\usepackage{amsmath}
\usepackage{amssymb}
\usepackage{mathtools}
\usepackage{xfrac}

\newcommand{\tightmathT}{%
    \setlength{\abovedisplayskip}{0pt}
    \setlength{\abovedisplayshortskip}{0pt}
}
\newcommand{\tightmathB}{%
    \setlength{\belowdisplayskip}{0pt}
    \setlength{\belowdisplayshortskip}{0pt}
}
\newcommand{\tightmath}{%
    \tightmathT%
    \tightmathB%
}

\DeclareMathOperator{\CDiox}{\mathrm{CO}_2}

\begin{document}

\section{Ideal gas mixer}

\subsection{Utility functions}

\begin{multicols}{2}
    \noindent
    For emptying the input pipe $I$ into the furnace
    \columnbreak
    \begin{minted}{text}
        fillFurnace:
        yield
        s Furnace SettingInput 100
        l x IAnalyzer TotalMoles
        brgtz -3
        s Furnace SettingInput 0
        j ra
    \end{minted}
\end{multicols}

\begin{multicols}{2}
    \noindent
    And for emptying the furnace into the filtration system, which filters all
    $\CDiox$ back into the input pipe.
    \columnbreak
    \begin{minted}{text}
        emptyFurnace:
        yield
        s Furnace SettingOutput 100
        l x Furnace TotalMoles
        brgtz -3
        s Furnace SettingOutput 0
        j ra
    \end{minted}
\end{multicols}

\subsection{Mixing algorithm}

\begin{multicols}{2}
    \raggedcolumns
    \noindent
    Let $n_R=0$ be the moles to remove, and calculate constant terms.
    \tightmathB%
    \begin{flalign*}
        e_T &= t_T n_T & \\
        n_{T-F} &= n_T-n_F & \\
        t_{H-C} &= t_H-t_C &
    \end{flalign*}
    \columnbreak

    \begin{minted}{text}
        move nR 0
        mul eT tT nT
        sub dn nT nF
        sub dt tH tC
    \end{minted}
\end{multicols}

\subsection{Calculate input moles $n_I=n_H+n_C$ given $n_R$}\label{alg:input}

\begin{multicols}{2}
    \raggedcolumns
    \noindent
    Calculate total moles input
    \tightmathB
    \begin{flalign*}
        n_I = n_{T-F}+n_R &&
    \end{flalign*}
    \columnbreak

    \begin{minted}{text}
        calculateInputMoles:
        add nI dn nR
    \end{minted}
\end{multicols}
\begin{multicols}{2}
    \raggedcolumns
    \noindent
    Calculate temperature input:
    \tightmathB
    \begin{flalign*}
        t_I = \frac{e_T-t_F(n_F-n_R)}{n_I} &&
    \end{flalign*}
    \columnbreak

    \begin{minted}{text}
        sub x nF nR
        mul x tF x
        sub tI eT x
        div tI tI nI
    \end{minted}
\end{multicols}
\begin{multicols}{2}
    \raggedcolumns
    \noindent
    Calculate moles from hot/cold source:
    \tightmathB
    \begin{flalign*}
        n_H = \frac{n_I(t_I-t_C)}{t_{H-C}},\quad
        n_C = n_I-n_H &&
    \end{flalign*}
    \columnbreak

    \begin{minted}{text}
        sub nH tI tC
        mul nH nI nH
        div nH nH dt
        sub nC nI nH
        j ra
    \end{minted}
\end{multicols}

\subsection{Searching for $n_R$}

\begin{multicols}{2}
    \noindent
    For $n_R$ to be valid, $n_H$ and $n_C$ must both be positive.
    \columnbreak
    \begin{minted}{text}
        checkInputMoles:
        min x nH nC
        sgtz x x # x = nH > 0 and nC > 0
        j ra
    \end{minted}
\end{multicols}
\begin{multicols}{2}
    \noindent
    For $n_R=0$, if this is the case we are done.
    \columnbreak
    \begin{minted}{text}
        move nR 0
        jal calculateInputMoles
        jal checkInputMoles
        beq x 1 compose # nR=0, nH>0, nC>0: skip
    \end{minted}
\end{multicols}
\noindent
Else, we begin a binary search of the domain $[0,n_F]$
for a satisfactory value of $n_R$, each time recalculating and rechecking the
input moles via \ref{alg:input}.
Beginning with $n_R=\sfrac{n_F}{2}$, we bisect $i$ times.
\begin{minted}{text}
    define BisectionIterations 20
    # ...
    move i BisectionIterations
    div nR nF 2
    div step nF 4
    jal calculateInputMoles
    jal checkInputMoles
    select x x 1 -1
    mul x x step
    add nR nR x # nR += (nH>0 && nC>0) ? step : -step
    div step step 2
    sub i i 1
    brgtz i -7
\end{minted}

\subsection{Removing $n_R$ and composing $I$}\label{alg:composing}

\begin{multicols}{2}
    \raggedcolumns
    \noindent
    Having found satisfactory $n_R$, $n_H$ and $n_C$, we first remove $n_R$
    moles from the furnace (or remove none when $n_R$ is zero).
    The furnace, containing only $\CDiox$, is emptied into the input pipe $I$.
    Moles are removed, via a volume pump, until the total moles in the pipe
    $n_I$ is less-than or equal to the initial number of moles $n_F$ minus the
    number of moles to remove $n_R$. Then this potentially reduced amount is
    reinserted into the furnace.
    \columnbreak
    \begin{minted}{text}
        jal emptyFurnace
        sub nF nF nR
        yield
        s InputDumpPump On 1
        l x IAnalyzer TotalMoles
        brgt x nF -3
        s InputDumpPump On 0
        jal fillFurnace
    \end{minted}
\end{multicols}

%\begin{multicols}{2}
\begin{parcolumns}{2}
    \colchunk[1]{%
        We then compose the $I$ mixture by first adding $H$ moles via a volume pump
        while the moles in $I$ is less-than to $n_H$, then doing the
        same for $C$ until moles in $I$ is less-than to $n_I$.
    }\colchunk[2]{%
        \begin{minted}{text}
            alias HPump d0
            alias CPump d1
            alias HAnalyzer d2
            alias CAnalyzer d3

            alias nH r0
            alias hC r1
            alias i r2

            move i 0

            fillUntil:
            yield
            s dr2 On 1
            add i i 2
            l x dr3 TotalMoles
            sub i i 2
            brlt x rr2 - 3
            s dr0 On 0
            add i i 1
        \end{minted}
    }
\end{parcolumns}

\end{document}
